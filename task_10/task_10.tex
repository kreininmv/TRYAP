\documentclass[a4paper,14pt]{article} % тип документа
%\documentclass[14pt]{extreport}
\usepackage{extsizes} % Возможность сделать 14-й шрифт


\usepackage{geometry} % Простой способ задавать поля
\geometry{top=20mm}
\geometry{bottom=25mm}
\geometry{left=15mm}
\geometry{right=15mm}

\setcounter{section}{0}

%%%Библиотеки
%\usepackage[warn]{mathtext}
%\usepackage[T2A]{fontenc} % кодировка
\usepackage[utf8]{inputenc} % кодировка исходного текста
\usepackage[english,russian]{babel} % локализация и переносы
\usepackage{caption}
\usepackage{listings}
\usepackage{amsmath,amsfonts,amssymb,amsthm,mathtools}
\usepackage{wasysym}
\usepackage{graphicx}%Вставка картинок правильная
\usepackage{float}%"Плавающие" картинки
\usepackage{wrapfig}%Обтекание фигур (таблиц, картинок и прочего)
\usepackage{fancyhdr} %загрузим пакет
\usepackage{lscape}
\usepackage{xcolor}
\usepackage{dsfont}
%\usepackage{indentfirst}
\usepackage[normalem]{ulem}
\usepackage{hyperref}




%%% DRAGON STUFF
\usepackage{scalerel}
\usepackage{mathtools}

\DeclareMathOperator*{\myint}{\ThisStyle{\rotatebox{25}{$\SavedStyle\!\int\!\!\!$}}}

\DeclareMathOperator*{\myoint}{\ThisStyle{\rotatebox{25}{$\SavedStyle\!\oint\!\!\!$}}}

\usepackage{scalerel}
\usepackage{graphicx}
%%% END 

%%%Конец библиотек

\newcommand{\drawsome}[1]{            % Для быстрой вставки картинок
    \begin{figure}[h!]
            \centering
            \includegraphics[scale=0.7]{#1}
            \label{fig:first}
    \end{figure}
}
\newcommand{\drawsomemedium}[1]{
    \begin{figure}[h!]
            \centering
            \includegraphics[scale=0.45]{#1}
            \label{fig:first}
    \end{figure}
}
\newcommand{\drawsomesmall}[1]{
    \begin{figure}[h!]
            \centering
            \includegraphics[scale=0.3]{#1}
            \label{fig:first}
    \end{figure}
}

%%%Настройка ссылок
\hypersetup
{
colorlinks=true,
linkcolor=blue,
filecolor=magenta,
urlcolor=blue
}
%%%Конец настройки ссылок


%%%Настройка колонтитулы
	\pagestyle{fancy}
	\fancyhead{}
	\fancyhead[L]{Домашнее задание}
	\fancyhead[R]{Крейнин Матвей, группа Б05--005}
	\fancyfoot{}
    \fancyfoot[C]{\thepage}
    \fancyfoot[R]{ТРЯП}
%%%конец настройки колонтитулы


\begin{document}
%%%%Начало документа%%%%

\section{Задание 10}
\subsection{Задача 1}
\textbf{а)}
\underline{Ответ:} Не является LL(k)-грамматикой при любом k
\begin{enumerate}
    \item $S \longrightarrow^*_l Aa^kb \longrightarrow_l a^{k+1}b, \omega = \varepsilon, \alpha = a^kb, \beta = a$
    \item $S \longrightarrow^*_l Aa^kb \longrightarrow_l Aa^{k+1}b, \omega = \varepsilon, \alpha = a^kb, \gamma = Aa$
\end{enumerate}
$FIRST_k(\beta \alpha) \cup FIRST_k(\gamma \alpha) \neq \emptyset$ для любого k.
\newline
\textbf{б)} \underline{Ответ:} Является LL(k)-грамматикой при $k \geqslant 2$.
\begin{enumerate}
    \item $S \longrightarrow^*_l a^kAb \longrightarrow_l a^{k}ab, \omega = a^k, \alpha = b, \beta = a$
    \item $S \longrightarrow^*_l a^kAb \longrightarrow_l a^kaAb, \omega = a^k, \alpha = b, \gamma = aA$
\end{enumerate}
$FIRST_k(\beta \alpha) \cup FIRST_k(\gamma \alpha) = \emptyset$  $k \geqslant 2$.
\newline
\textbf{в)} \underline{Ответ:} Не является LL(k)-грамматикой для любого k.
\begin{enumerate}
    \item $S \longrightarrow^*_l aBBb \longrightarrow_l aabBb, \omega = a, \alpha = Bb, \beta = ab$
    \item $S \longrightarrow^*_l aBBb \longrightarrow_l a \varepsilon Bb, \omega = a, \alpha = Bb, \gamma = \varepsilon$
\end{enumerate}
Используя индукцию по k получим, что $FIRST_k(\beta \alpha) \cup FIRST_k(\gamma \alpha) \neq \emptyset$, при $k = 1$ и $k = 2$ возьмём $B \longrightarrow ab$; при $k \geqslant 3$, возьмём $B \longrightarrow \varepsilon$ в первом и случае и $B \longrightarrow ab во втором$:
\newline
$FIRST_k(abBb) \cup FIRST_k(Bb) \neq \emptyset$, Грамматика не является LL(k для любого k
\newline
\textbf{г)} \underline{Ответ:} Не является LL(k)-грамматикой для любого k
\begin{enumerate}
    \item $S \longrightarrow^*_l aaBB \longrightarrow_l aaaBBB, \omega = aa, \alpha = B, \beta = aBB$
    \item $S \longrightarrow^*_l aaBB \longrightarrow_l aabB, \omega = aa, \alpha = B, \gamma = b$
\end{enumerate}
$FIRST_k(\beta \alpha) \cup FIRST_k(\gamma \alpha) \neq \emptyset$, т.к. содержит $\omega = a^k$. Поэтому грамматика не будет LL(k)-грамматикой для любого k.
\newline
\textbf{д)} \underline{Ответ:} Является LL(k)-грамматикой для любого k.
\begin{enumerate}
    \item $S \longrightarrow^*_l aaBB \longrightarrow_l aaaBBB, \omega = aa, \alpha = B, \beta = aBB$
    \item $S \longrightarrow^*_l aaBB \longrightarrow_l aabB, \omega = aa, \alpha = B, \gamma = b$
\end{enumerate}
Видно, что $FIRST_k(\beta \alpha) \cup FIRST_k(\gamma \alpha) = \emptyset$, поэтому это LL(k)-грамматика.

\subsection{Задача 2}
Получим грамматику: $S \rightarrow Ab, A \rightarrow aA^{'}, A^{'} \rightarrow a | \varepsilon$, воспользовавашись алгоритмом удаления правого ветвления.


\begin{tabular}{|l|lll|lll|}
    \hline
         & \multicolumn{3}{l|}{FIRST(X)}                                                                                                 & \multicolumn{3}{l|}{FOLLOW(X)}                                                                     \\ \hline
    $X  $ & \multicolumn{1}{l|}{S}                        & \multicolumn{1}{l|}{A}                        & $A^{'}$      & \multicolumn{1}{l|}{S}  & \multicolumn{1}{l|}{A}           & $A^{'}$ \\ \hline
    $F_0$ & \multicolumn{1}{l|}{$\emptyset$} & \multicolumn{1}{l|}{$\emptyset$} & $\emptyset$      & \multicolumn{1}{l|}{\$} & \multicolumn{1}{l|}{$\emptyset$} & $\emptyset$ \\ \hline
    $F_1$ & \multicolumn{1}{l|}{a}                        & \multicolumn{1}{l|}{a}                        & a, $\varepsilon$ & \multicolumn{1}{l|}{\$} & \multicolumn{1}{l|}{b}                        & $\emptyset$ \\ \hline
    $F_2$ & \multicolumn{1}{l|}{a}                        & \multicolumn{1}{l|}{a}                        & a, $\varepsilon$ & \multicolumn{1}{l|}{\$} & \multicolumn{1}{l|}{b}                        & b                        \\ \hline
\end{tabular}
\newline
Используем алгоритм построения анализатора:



\begin{tabular}{|l|l|l|l|}
    \hline
       & a                    & b                        & \$    \\ \hline
    S  & $S \longrightarrow Ab   $ & error                    & error \\ \hline
    A  & $A \longrightarrow aA^{'}$ & error                    & error \\ \hline
    $A^{'}$ & $A^{'} \longrightarrow a  $ & $A^{'} \longrightarrow \varepsilon$ & error \\ \hline
\end{tabular}
\newline
Получим LL(k)-грамматику, т.к. в каждой ячейку получилось записано не больше одного правила.

\subsection{Задача 3}
Используем алгоритм удаления левой рекурссии:
\newline
$S \longrightarrow baaA | bab A$
\newline
$A \longrightarrow aA | bA | a | b$
\newline 
Удалим правое ветвление:
\newline
$S \longrightarrow baS^{'}$
\newline
$S^{'} \longrightarrow aA | bA$
\newline
$A \longrightarrow aA | bA | \varepsilon$
\newline
Строим таблицу FIRST и FOLLOW:


\begin{tabular}{|l|lll|lll|}
    \hline
          & \multicolumn{3}{l|}{FIRST(X)}                                                                                                 & \multicolumn{3}{l|}{FOLLOW(X)}                                                                     \\ \hline
    $X  $ & \multicolumn{1}{l|}{S}           & \multicolumn{1}{l|}{$S^{'}$}     & $A$                 & \multicolumn{1}{l|}{S}  & \multicolumn{1}{l|}{$S^{'}$}     & $A$         \\ \hline
    $F_0$ & \multicolumn{1}{l|}{$\emptyset$} & \multicolumn{1}{l|}{$\emptyset$} & $\emptyset$         & \multicolumn{1}{l|}{\$} & \multicolumn{1}{l|}{$\emptyset$} & $\emptyset$ \\ \hline
    $F_1$ & \multicolumn{1}{l|}{b}           & \multicolumn{1}{l|}{a, b}        & a, b, $\varepsilon$ & \multicolumn{1}{l|}{\$} & \multicolumn{1}{l|}{\$}          & $\emptyset$ \\ \hline
    $F_2$ & \multicolumn{1}{l|}{b}           & \multicolumn{1}{l|}{a, b}        & a, b, $\varepsilon$ & \multicolumn{1}{l|}{\$} & \multicolumn{1}{l|}{\$}          & \$          \\ \hline
\end{tabular}
\newline
Построим анализатор:


\begin{tabular}{|l|l|l|l|}
    \hline
             & a                           & b                           & \$    \\ \hline
    S        & error                       & $S \longrightarrow baS^{'}$ & error \\ \hline
    $S^{'}$  & $S^{'} \longrightarrow aA$  & $S^{'} \longrightarrow bA$  & error \\ \hline
    A        & $A \longrightarrow aA  $    & $A \longrightarrow bA$      & $A \longrightarrow \varepsilon$ \\ \hline
\end{tabular}
\newline
Продемнострируем работа анализатора на слове baab:
$S \longrightarrow baS^{'} \longrightarrow baaA \longrightarrow baab$

\subsection{Задача 5}
Видно, что это грамматика языка Дика. А язык Дика задаётся грамматикой:
$S \longrightarrow (S)S | \varepsilon$
\newline Получили мы эту грамматику, используя алгоритм удаления левой рекурсии и правого ветвления.
\begin{enumerate}
    \item $S \longrightarrow^*_l (S)S \longrightarrow_l ((S)S)S, \omega = (, \alpha = )S, \beta = (S)S $
    \item $S \longrightarrow^*_l (S)S \longrightarrow_l (\varepsilon)S, \omega = (, \alpha =)S, \gamma = \varepsilon$
\end{enumerate}
$FIRST_1[(S)S)S] \cup FIRST_1[)S] = \emptyset$, $FIRST_1[S] \cup FOLLOW_1[S] = \emptyset$
Видно, что после S может быть только <<)>>, поэтому последнее равенство верно.

\subsection{Задача 6}
Пусть у нас есть праволинейная грамматика: $G = \langle \{S, A, B\}, \{a, b, \$ \}, P, S\rangle$, где P: $S \longrightarrow A, A \longrightarrow ba | b$
\newline
Это не будет LL(1)-грамматикой, т.к. нет правил с $\varepsilon$:
$FIRST_1(ba) \cup FIRST_1(b) = {b} \neq \emptyset$
\subsection{Задача 7}
\underline{Ответ:} Да, верно.
\newline
$G = \langle \{S, A, B\}, \{a, b, \$ \}, P, S \rangle$, P: $S \longrightarrow AB, A \longrightarrow \varepsilon, B \longrightarrow \varepsilon$
\newline
Построим анализатор (очевидно, что таблица first и follow будет содержать только $\varepsilon$ и \$ соотвественно):


\begin{tabular}{|l|l|l|l|}
    \hline
       & a     & b     & \$                              \\ \hline
    S  & error & error & $S \longrightarrow AB$          \\ \hline
    A  & error & error & $A \longrightarrow \varepsilon$ \\ \hline
    B  & error & error & $B \longrightarrow \varepsilon$ \\ \hline
\end{tabular}
\newline
Видим, что каждой ячейке таблицы соотвествует не больше одного правила вывода, из чего делаем вывод, что грамматика является LL(1)-грамматикой.
\subsection{Задача 8}
\underline{Ответ:} Нет, неверно.
\newline
Рассмотрим грамматику $G = \langle \{S, A, B\}, \{a, b, \$\}, P, S\rangle$, где P: $S \longrightarrow AB, A \longrightarrow \varepsilon, B \longrightarrow \varepsilon$
\newline
Видно, что $FOLLOW(A) \cup FOLLOW(B) = \{ \$ \}$, но грамматика G является LL(1)-грамматикой, было доказано в предыдущей задаче.

\end{document}
