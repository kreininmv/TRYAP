\documentclass[a4paper,14pt]{article} % тип документа
%\documentclass[14pt]{extreport}
\usepackage{extsizes} % Возможность сделать 14-й шрифт


\usepackage{geometry} % Простой способ задавать поля
\geometry{top=25mm}
\geometry{bottom=35mm}
\geometry{left=20mm}
\geometry{right=20mm}

\setcounter{section}{0}

%%%Библиотеки
%\usepackage[warn]{mathtext}
%\usepackage[T2A]{fontenc} % кодировка
\usepackage[utf8]{inputenc} % кодировка исходного текста
\usepackage[english,russian]{babel} % локализация и переносы
\usepackage{caption}
\usepackage{listings}
\usepackage{amsmath,amsfonts,amssymb,amsthm,mathtools}
\usepackage{wasysym}
\usepackage{graphicx}%Вставка картинок правильная
\usepackage{float}%"Плавающие" картинки
\usepackage{wrapfig}%Обтекание фигур (таблиц, картинок и прочего)
\usepackage{fancyhdr} %загрузим пакет
\usepackage{lscape}
\usepackage{xcolor}
\usepackage{dsfont}
%\usepackage{indentfirst}
\usepackage[normalem]{ulem}
\usepackage{hyperref}




%%% DRAGON STUFF
\usepackage{scalerel}
\usepackage{mathtools}

\DeclareMathOperator*{\myint}{\ThisStyle{\rotatebox{25}{$\SavedStyle\!\int\!\!\!$}}}

\DeclareMathOperator*{\myoint}{\ThisStyle{\rotatebox{25}{$\SavedStyle\!\oint\!\!\!$}}}

\usepackage{scalerel}
\usepackage{graphicx}
%%% END 

%%%Конец библиотек

%%%Настройка ссылок
\hypersetup
{
colorlinks=true,
linkcolor=blue,
filecolor=magenta,
urlcolor=blue
}
%%%Конец настройки ссылок


%%%Настройка колонтитулы
	\pagestyle{fancy}
	\fancyhead{}
	\fancyhead[L]{Домашнее задание}
	\fancyhead[R]{Крейнин Матвей, группа Б05-005}
	\fancyfoot{}
    \fancyfoot[C]{\thepage}
    \fancyfoot[R]{ТРЯП}
%%%конец настройки колонтитулы



\begin{document}
%%%%Начало документа%%%%

\section{Задание 1}
\subsection{Задача 1}

$\textbf{1.}0$ Нет, т.к. b $\notin \{1, 2, 3\}$ или же $1 \notin \{a, b\}$.
\newline
$\textbf{2.}$ $|A \times B| = |A| \cdot |B|$, т.к. мы можем выбрать |A| элементов из первого множества 
и |B| элементов из второго множества, сл-но всего элементов в множестве $|A \times B|$ будет $|A| \cdot |B|$
\newline
$\textbf{3.}$ Это получится $\invdiameter$, т.к. пустое множество не содержит в себе элементов, а значит 
и в множестве $\mathds{N} \times \invdiameter$ не будет элементов, сл-но оно пустое.

\subsection{Задача 2}
$\textbf{a)}$ Да, верно, т.к. пустое слово содержится в любом непустом множестве.
\newline
$\textbf{b)}$ Да, верно, т.к. пустое множество содержитс в непустом множестве.

\subsection{Задача 3}
Регулярное выражение: $X^{+}X^{+}$, т.к. у нас будут все слова в этом множестве вида $a^{n}a^{m}$, где $m, n \in \mathds{N}$ \& $m, n>1$

\subsection{Задача 4}
\textbf{а)} $a^{*}b^{*}(a|b)^{+}$ - 
\newline
\textbf{b)} $(b^{*}a^{*})$ ab $(b^{*}a^{*})$ Я исхожу, из доказанного регулярного выражения в пункте в), то есть $b^{*}a^{*}$ 
\newline
\textbf{в)} $b^{*}a^{*}$, в этом языке не может идти ни одной буквы b, после буквы a, иначе это слово содержит в себе подслово ab, сл-но оно состоит из нескольких (или нуля) подряд идущих букв b или нескольких (или нуля) подряд идущих букв a.
Это и описывает заданное регулярное выражение.
\subsection{Задача 5}
$1.$ Предлагаю сначала доказать включение L в R, у нас нет слов, содержащих 3 букв b подряд в языке L, база, у нас есть только $\mathcal{E}, b, bb$, база доказана таких слов нет.
Теперь предположим, что их нет и на n-1, составлении при помощи правила 2, тогда при составлении слова на n шаге, мы можем составить слово из букв ax, bax и bbax, но в x нет последовательности 
из 3-х подряд b, а при составлении не возникает трех подряд идущих b, т.к. их от других b отделяет буква a. 

Теперь нужно доказать, что из языка L можно составить любое слово, не содержащее

\end{document}