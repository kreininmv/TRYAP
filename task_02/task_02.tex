\documentclass[a4paper,14pt]{article} % тип документа
%\documentclass[14pt]{extreport}
\usepackage{extsizes} % Возможность сделать 14-й шрифт


\usepackage{geometry} % Простой способ задавать поля
\geometry{top=25mm}
\geometry{bottom=35mm}
\geometry{left=20mm}
\geometry{right=20mm}

\setcounter{section}{0}

%%%Библиотеки
%\usepackage[warn]{mathtext}
%\usepackage[T2A]{fontenc} % кодировка
\usepackage[utf8]{inputenc} % кодировка исходного текста
\usepackage[english,russian]{babel} % локализация и переносы
\usepackage{caption}
\usepackage{listings}
\usepackage{amsmath,amsfonts,amssymb,amsthm,mathtools}
\usepackage{wasysym}
\usepackage{graphicx}%Вставка картинок правильная
\usepackage{float}%"Плавающие" картинки
\usepackage{wrapfig}%Обтекание фигур (таблиц, картинок и прочего)
\usepackage{fancyhdr} %загрузим пакет
\usepackage{lscape}
\usepackage{xcolor}
\usepackage{dsfont}
%\usepackage{indentfirst}
\usepackage[normalem]{ulem}
\usepackage{hyperref}




%%% DRAGON STUFF
\usepackage{scalerel}
\usepackage{mathtools}

\DeclareMathOperator*{\myint}{\ThisStyle{\rotatebox{25}{$\SavedStyle\!\int\!\!\!$}}}

\DeclareMathOperator*{\myoint}{\ThisStyle{\rotatebox{25}{$\SavedStyle\!\oint\!\!\!$}}}

\usepackage{scalerel}
\usepackage{graphicx}
%%% END 

%%%Конец библиотек

%%%Настройка ссылок
\hypersetup
{
colorlinks=true,
linkcolor=blue,
filecolor=magenta,
urlcolor=blue
}
%%%Конец настройки ссылок


%%%Настройка колонтитулы
	\pagestyle{fancy}
	\fancyhead{}
	\fancyhead[L]{Домашнее задание}
	\fancyhead[R]{Крейнин Матвей, группа Б05-005}
	\fancyfoot{}
    \fancyfoot[C]{\thepage}
    \fancyfoot[R]{ТРЯП}
%%%конец настройки колонтитулы



\begin{document}
%%%%Начало документа%%%%

\section{Задание 2}
\subsection{Задача 1}

$\textbf{1.}$ R = $(bb|b|\mathcal{E})$$(a^{+}(bb|b|\mathcal{E}))^{*}$
Теперь доказательство корректности регулярного выражения, пусть n = 1, для слова a, регялрное выражение считается корректно, в $(bb|b|\mathcal{E})$ будет $\mathcal{E}$
потом считается a из $a^{+}$ потом считается $\mathcal{E}$ и на этом закончится (для слова b получается почти аналогично), для aa, ab, bb, ba - всё очевидно, в первом случае считается две буквы а из второго скобки, для ab считается a потом b из второй скобки, для bb считается две буквы b из первой скобки, для ba считается буква b из первой скобки, потом буква a из второй.
База доказана

Предположим, что это верно для слова длины n, тогда рассмотрим слово длины n+1.
n > 2. Тогда получается, что первая скобка считалась. У нас слово длины n при добавлении буквы а может оканчиваться на любую букву, считывании произойдёт, т.к. во второй скобке считается просто еще одна буква a.
В случае же приписывании буквы b слово длины n может оканчиваться на букву a, либо на одну b. В случае буквы a считается эта буква во второй скобке, в случае b произойдет считывание двух букв b, вместо одной буквы b.
Т.к. до этого по индукции было предположено, что трех подряд букв нет. То и в новом построенном слове такой буквы не будет. Если же слово длины n будет оканчиваться на две буквы bb и новая буква b, то РВ R его не считает, т.к. в указанном регулярном выражении не может быть трех букв b подряд, их разделяет хотя бы одна буква a.  

\subsection{Задача 2}
$\textbf{1.}$ Автомат $\mathcal{A}$: $Q = \{q_0, q_1, q_2\}$, $\Sigma = \{0, 1\}$, $q_0 = q_0$, F = $q_1$
\begin{tabular}{ | l | l | l | }
    \hline
    $\delta:$ & 0       & 1     \\ \hline
    $q_0$ & $q_0$   & $q_1$ \\
    $q_1$ & $q_2$   & $q_0$ \\
    $q_2$ & $q_1$   & $q_2$ \\
    \hline
    \end{tabular}

\vspace{10mm}

Автомат $\mathcal{B}$: $Q = \{q_0, q_1, q_2\}$, $\Sigma = \{0, 1\}$, $q_0 = q_0$, F = $q_1$
\begin{tabular}{ | l | l | l | }
    \hline
    $\delta:$ & 0       & 1             \\ \hline
    $q_0$ & $q_0$       & $q_1$         \\
    $q_1$ & $q_2, q_0$  & не определен  \\
    $q_2$ & $q_1$       & $q_2$         \\
    \hline
    \end{tabular}

\newpage
$\textbf{2.}$ Автомат $\mathcal{A}$ является детерменированным, это видно из таблицы переходов, т.к. там однозначно определен из каждого состояния в каждое.

Автомат $\mathcal{B}$ не является детерменированным, т.к. у него не определен однозначно переход из состояния $q_1$ по букве 0 (может быть $q_2$, а может быть $q_0$), переход по букве 1 вообще не определен.

$\textbf{3.}$ При старте автомат $\mathcal{A}$ находится в состояние $q_0$. По букве 0 автомат останется в состоянии $q_0$, по букве 1 автомат перейдёт в состояние $q_1$, по букве 1 перейдёт обратно в состояние $q_0$,
по букве 0 и 0 останется в состоянии $q_0$, потом по букве 1 перейдёт в состоянии $q_1$. Состояние $q_1$ является принимающим состоянием, поэтому слово $\omega \in \mathcal{L(A)}.=$

$\textbf{4.}$
\end{document}